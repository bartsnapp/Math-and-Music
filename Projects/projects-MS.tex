\documentclass[12pt,noauthor,nooutcomes,hints,instructornotes]{ximera}%% SEE: https://www.overleaf.com/project/626c378b4e3e27903576f685
\author{}

%% for par indent/spacing
%\usepackage[skip=20pt plus1pt, indent=10pt]{parskip}
%% we have setspace loaded below.. maybe that is better.




\graphicspath{  
{./}
{./rhythm/}
{./timeSignature/}
{./pitch/}
{./modularNotes/}
{./waves/}
{./superposition/}
{./stars/}
}

%\input Starburst.fd
%\newcommand*\initfamily{\usefont{U}{Starburst}{xl}{n}}
%\et\othermaketitle\maketitle
%\enewcommand{\maketitle}{ %
%\pagestyle{empty}
%\theauthor
%{\initfamily \scalebox{10}{Math}\\[10pt]\scalebox{6}{and} \\[10pt] \scalebox{10}{music}}
%\newpage
%\pagestyle{main} %switch to main pagestyle, just like ximera documents.
%\let\maketitle\othermaketitle % renew maketitle to usual definition.
%}

\newcommand{\BAMM}{\textbf{\textsf{BAMM}}}

\renewcommand{\link}[2][]{\href{#2}{#1}}

%% to fix multicol in triplets in beatsAndTime
\usepackage{array}
\newcolumntype{C}[1]{>{\centering\let\newline\\\arraybackslash\hspace{0pt}}p{#1}}




%% for multiple staffs same line mod notes
\usepackage{multicol}
\usepackage{xfrac}


%% Larger sharp and flat
%\let\tmpsharp\sharp
%\renewcommand{\sharp}{{\mbox{\normalsize$\tmpsharp$}}}
%\let\tmpflat\flat
%\renewcommand{\flat}{{\mbox{\normalsize$\tmpflat$}}}




\newcommand{\Z}{\mathbb Z}
\newcommand{\lto}{\mathop{\longrightarrow\,}\limits} %% labeled to

\renewcommand{\gcd}{\mathrm{GCD}}
\newcommand{\lcm}{\mathrm{LCM}}

\newcommand{\answerbox}{\raisebox{-4ex}{\fbox{\hspace{1.25in}\rule[.7in]{0pt}{0pt}}}}
\newcommand{\smallanswerbox}{\raisebox{-4ex}{\fbox{\hspace{.5in}\rule[.7in]{0pt}{0pt}}}}


%% for hertz and cents
\usepackage{musicography}
\usepackage{musixtex}
\input{musixper}
\def\threedp#1{\pgfmathparse{#1}\expandafter\Threedp\pgfmathresult0000@}
\def\Threedp#1.#2#3#4#5@{#1.#2}

%%%%% End herz cents
%%%%%

%% for annotated bib
\let\oldbibliography\thebibliography%% to compact bib
\renewcommand{\thebibliography}[1]{%
  \oldbibliography{#1}%
  \setlength{\itemsep}{0pt}%
}
\renewcommand\refname{} %% no name needed!
\usepackage{setspace} %% for tighter annotate
\NewEnviron{annotate}{\vspace{-.1cm}\small \begin{spacing}{0.1} \itshape \BODY \end{spacing}\vspace{1cm}}




\usepackage[tikz]{mdframed}
\mdfdefinestyle{OutcomeStyle}{backgroundcolor=none,leftmargin=2cm,rightmargin=2cm,linecolor=black,roundcorner=5pt,
, font=\small\sffamily,}%font=\wedn\bfseries\upshape,}
\newenvironment{listOutcomes}{\begin{mdframed}[style=OutcomeStyle]After answering the following questions, students should be able to:\begin{itemize}}{\end{itemize}\end{mdframed}}
\newenvironment{listSectionOutcomes}{\begin{mdframed}[style=OutcomeStyle]After completing the following journal entries, students should be able to:\begin{itemize}}{\end{itemize}\end{mdframed}}





\usepackage{kmath,kerkis} % The order of the packages matters; kmath changes the default text font
\usepackage{fullpage}
\usepackage[T1]{fontenc}

















\usepackage{wasysym,qtree}



%% TO HERE!!
\def\freqbarno{9999}% no bar numbers




%https://tex.stackexchange.com/questions/266560/drawing-music-staves-with-rule-loop
\newcommand{\staff}[2][5]{%[# of lines per staff]{# of staves}
    \begingroup
    \parindent0pt
    \noindent
    \foreach \y in {1,...,#2}{%
        \foreach \x in {1,2,...,#1}{%
            \rule{\textwidth}{0.4pt}% {<staff width>}{<line thickness>}
            \par
            \nointerlineskip
            \vskip6pt% Distance between the lines
        }%
        \vspace{0.25in}% Distance between staves
    }%
    \endgroup
}



%% thm like environments
\let\question\relax
\let\endquestion\relax
%\newtheoremstyle{QuestionStyle}{\topsep}{\topsep}%%% space between body and thm
%		{}                      %%% Thm body font
%		{}                              %%% Indent amount (empty = no indent)
%		{\bfseries}            %%% Thm head font
%		{)}                              %%% Punctuation after thm head
%		{ }                           %%% Space after thm head
%		{\thmnumber{#2}\thmnote{ \bfseries(#3)}}%%% Thm head spec
%\theoremstyle{QuestionStyle}
%\newtheorem{question}{}



%% Our custom question environment
\newcounter{question}[problem]
\newenvironment{question}{\refstepcounter{question}\ifhandout\newpage\fi
\par\medskip
   \noindent \textbf{\thequestion)} \rmfamily}{\medskip}

\newcounter{project}[problem]
\newenvironment{project}{\refstepcounter{project}
\par\medskip
   \noindent \textbf{\theproject)} \rmfamily}{\medskip}



\newcommand{\answerspace}{\ifhandout
\hspace{0pt}
\vspace{1in}
\hspace{0pt}
\fi}

\newcommand{\answerlines}{\ifhandout
  \rule{\textwidth}{.5pt}
  
  \rule{\textwidth}{.5pt}
  
  \rule{\textwidth}{.5pt}\fi
}

\newcommand{\answermusic}{\ifhandout
  \rule{\textwidth}{.5pt}\\[-7pt]
  \rule{\textwidth}{.5pt}\\[-7pt]
  \rule{\textwidth}{.5pt}\\[-7pt]
  \rule{\textwidth}{.5pt}\\[-7pt]
  \rule{\textwidth}{.5pt}\fi
}




%% see https://tex.stackexchange.com/questions/508619/fill-page-with-lines
%% see https://tex.stackexchange.com/questions/266560/drawing-music-staves-with-rule-loop
%% \staff{5}
%% end question


\newcommand{\timesig}[2]{\scalebox{.7}{\raisebox{-1ex}{\meterfrac{#1}{#2}}}}


\newcommand{\stringstar}[2]{\left\{\frac{#1}{#2}\right\}}




\makeatletter

\newcommand\frontstyle{%
  \def\activitystyle{activity-chapter}
  \def\maketitle{%
                {\flushleft\small\sffamily\bfseries\@pretitle\par\vspace{-1.5em}}%
                {\flushleft\LARGE\sffamily\bfseries\@title \par }%3
                {\vskip .6em\noindent\textit\theabstract\setcounter{question}{0}\setcounter{sectiontitlenumber}{0}}%
                    \par\vspace{2em}
                    \phantomsection\addcontentsline{toc}{section}{\textbf{\@title}}%
                     \setcounter{titlenumber}{0}}}



\newcommand\mmstyle{%
    \cleardoublepage
    \def\activitystyle{activity-section}
    \def\maketitle{%
    \addtocounter{titlenumber}{1}%
                {\flushleft\small\sffamily\bfseries\@pretitle\par\vspace{-1.5em}}%
                {\flushleft\LARGE\sffamily\bfseries\thetitlenumber\hspace{1em}\@title  \par }%
                {\vskip .6em\noindent\textit\theabstract\setcounter{question}{0}\setcounter{sectiontitlenumber}{0}}%
                    \par\vspace{2em}
                    \phantomsection\addcontentsline{toc}{section}{\thetitlenumber\hspace{1em}\textbf{\@title}}%
                     }}
\makeatother

\usepackage{fancyhdr}
\setlength{\headheight}{14pt} 

\fancyhf{}
\fancyhead[LE,RO]{\textsl{\thepage}}

\fancyfoot[LE,RO]{\includegraphics[width=1in]{bammLogo.png}}
\renewcommand{\headrulewidth}{0pt}




%% circledots
\usetikzlibrary{math}
\tikzset{
    pics/circledots/.style={
        code={
        \tikzmath{\xx = #1;\rr=2;}
        \foreach \x in {-1,0,...,\xx}
        {
        \draw[fill] ({\rr*sin(\x*360/\xx)},{\rr*cos(\x*360/\xx)}) circle (.1);
        }
        \draw[] (0,0) circle (\rr);
}}}





























%%%%% PIANO WITH STAFF 


\newcommand{\pianoWithStaff}{
%% from: https://tex.stackexchange.com/questions/136713/range-of-a-piano-keyboard-vs-staves
\pgfdeclarelayer{blacknotes}
\pgfsetlayers{main,blacknotes}
\tikzset{tight fit/.style={inner sep=0pt, outer sep=0pt}}
\begin{center}
\begin{tikzpicture}[scale=1,every node/.style={transform shape}]
  \def\lastnotenodename{clefs}
  \def\minNote{1}
  \def\maxNote{88}
\node [text width=1cm, tight fit] (clefs) at (0,0) {
    \begin{music}
        \instrumentnumber{1}
        \instrumentnumber{2}
        \nostartrule        
        \setstaffs1{1}
        \setstaffs2{1}  
        \setclef1{\bass}
        \setclef2{\treble}                                  
        \startextract
        \hskip2.0\elemskip
        \zendextract
    \end{music}
};

\foreach \note [
    evaluate={
        \n=int(mod(\note-1, 12));
        \octave=int((\note+8)/12);
        \t=int(floor((\note-1)/12)*7-7);
        \notename={"A","","B","C","","D","","E","F","","G",""}[\n];
        \tonicsolfa={"la","","si","so","","re","","mi","fa","","sol",""}[\n];
        \blacknote={0,1,0,0,1,0,1,0,0,1,0,1}[\n];
        \frequency=(2^((\note-49)/12))*440;}
] in {\minNote,...,\maxNote}{
    \ifnum\octave>3
        \tikzset{extract anchor/.style={anchor=south west, at=(\lastnotenodename.south east)}}
    \else
        \tikzset{extract anchor/.style={anchor=north west, at=(\lastnotenodename.north east)}}
    \fi
    \ifnum\blacknote=0
        \edef\notenodename{\notename_\octave}
        \node (\notenodename) [tight fit,text width=1cm, extract anchor/.try]  {%           
            \begin{music}
                \instrumentnumber{1}
                \instrumentnumber{2}
                \nostartrule        
                \setstaffs1{1}
                \setstaffs2{1}  
                \setclefsymbol1{\empty}
                \setclefsymbol2{\empty}     
                \setclef1{\bass}
                \setclef2{\treble}                      
                \startextract
                \transpose\t
                \hskip-1.5\elemskip         
                \ifnum\octave>3
                    \ifnum\octave>4
                        \Notes \nextinstrument \ql{\notename} \en       
                    \else
                        \Notes \nextinstrument \qu{\notename} \en                       
                    \fi
                \else
                    \ifnum\octave>2
                        \Notes \ql{\notename} \en
                    \else
                        \Notes \qu{\notename} \en
                    \fi
                \fi
                \zendextract
            \end{music}
        };
        \xdef\lastnotenodename{\notenodename}       
        \node [anchor=base] (sol-fa)  at (\notenodename |- 0,-3) {\tonicsolfa$_\octave$};

        \draw (\notenodename.south west |- 0,-4) rectangle ++(1, -4);
        \ifnum\n=2
        \node [xshift=0.2cm,rotate=90, font=\footnotesize, anchor=east] 
        at (\notenodename.north |- 0,-4) {\threedp\frequency};
        \fi
        \ifnum\n=3
        \node [xshift=-0.2cm,rotate=90, font=\footnotesize, anchor=east] 
        at (\notenodename.north |- 0,-4) {\threedp\frequency};
        \fi
        \ifnum\n=5
        \node [rotate=90, font=\footnotesize, anchor=east] 
        at (\notenodename.north |- 0,-4) {\threedp\frequency};
        \fi
        \ifnum\n=7
        \node [xshift=0.2cm,rotate=90, font=\footnotesize, anchor=east] 
        at (\notenodename.north |- 0,-4) {\threedp\frequency};
        \fi
        \ifnum\n=8
        \node [xshift=-0.2cm,rotate=90, font=\footnotesize, anchor=east] 
        at (\notenodename.north |- 0,-4) {\threedp\frequency};
        \fi
        \ifnum\n=10
        \node [xshift=0.0cm,rotate=90, font=\footnotesize, anchor=east] 
        at (\notenodename.north |- 0,-4) {\threedp\frequency};
        \fi

        
        \node [font=\footnotesize, anchor=south]  
            at (\notenodename.south |- 0,-8) {\note};
        \node [font=\footnotesize, anchor=south] 
            at (\notenodename.south |- 0,-8.5)  {$\notename_\octave$};
        \draw (\notenodename.south west |- sol-fa.south) 
            rectangle (\notenodename.south east |- 0,1.125); %0.125 by trial and error
    \else
                    \ifnum\n=1 %% A#--shift right
            \begin{pgfonlayer}{blacknotes}
              \fill ([xshift=-0.2cm]\lastnotenodename.north east |- 0,-4) rectangle ++(0.6, -2.5);
              \node  [xshift=0.1cm,rotate=90, text=white, font=\footnotesize, anchor=east]
                  at (\lastnotenodename.north east |- 0,-4) {\threedp\frequency};
            \end{pgfonlayer}
            \fi
            \ifnum\n=4 %% C# -- shift left
            \begin{pgfonlayer}{blacknotes}
              \fill ([xshift=-0.4cm]\lastnotenodename.north east |- 0,-4) rectangle ++(0.6, -2.5);
              \node  [xshift=-0.1cm,rotate=90, text=white, font=\footnotesize, anchor=east]
                  at (\lastnotenodename.north east |- 0,-4) {\threedp\frequency};
            \end{pgfonlayer}
            \fi
            \ifnum\n=6 %% D# -- shift right
            \begin{pgfonlayer}{blacknotes}
              \fill ([xshift=-0.2cm]\lastnotenodename.north east |- 0,-4) rectangle ++(0.6, -2.5);
              \node  [xshift=0.1cm,rotate=90, text=white, font=\footnotesize, anchor=east]
                  at (\lastnotenodename.north east |- 0,-4) {\threedp\frequency};
            \end{pgfonlayer}
            \fi
            \ifnum\n=9 %% F# -- shift left
            \begin{pgfonlayer}{blacknotes}
              \fill ([xshift=-0.4cm]\lastnotenodename.north east |- 0,-4) rectangle ++(0.6, -2.5);
              \node  [xshift=-0.1cm,rotate=90, text=white, font=\footnotesize, anchor=east]
                  at (\lastnotenodename.north east |- 0,-4) {\threedp\frequency};
            \end{pgfonlayer}
            \fi
            \ifnum\n=11 %% G# -- Don't move
            \begin{pgfonlayer}{blacknotes}
              \fill ([xshift=-0.3cm]\lastnotenodename.north east |- 0,-4) rectangle ++(0.6, -2.5);
              \node  [rotate=90, text=white, font=\footnotesize, anchor=east]
                  at (\lastnotenodename.north east |- 0,-4) {\threedp\frequency};
            \end{pgfonlayer}
            \fi
            \fi
}
\node [rotate=90] at (0,-6) {Frequency (Hz)};
\end{tikzpicture}
\end{center}%
}








\newcommand{\ttPiano}{%% from: https://tex.stackexchange.com/questions/136713/range-of-a-piano-keyboard-vs-staves
%%%%%%%%%%%%%%%%%%%%%%%%%%%%%%%%%%%%%%%%%%%%%%%%%%%%%%%%%%%%%%%%%%
%%%%%%%%%%%%%%%%%%%%%%%%%%%%%%%%%%%%%%%%%%%%%%%%%%%%%%%%%%%%%%%%%%
%% 
\pgfdeclarelayer{blacknotes}
\pgfsetlayers{main,blacknotes}
\tikzset{tight fit/.style={inner sep=0pt, outer sep=0pt}}

\begin{center}
\begin{tikzpicture}[scale=1,every node/.style={transform shape}]
  \def\lastnotenodename{clefs}
  \def\minNote{51}
  \def\maxNote{61}
\node [text width=1cm, tight fit] (clefs) at (0,-3) {
    \begin{music}
        \instrumentnumber{1}
        %\instrumentnumber{2}
        \nostartrule        
        \setstaffs1{1}
        %\setstaffs2{1}  
        %\setclef1{\bass}
        \setclef1{\treble}                                  
        \startextract
        \hskip2.0\elemskip
        \zendextract
    \end{music}
};

\foreach \note [
    evaluate={
        \n=int(mod(\note-1, 12));
        \octave=int((\note+8)/12);
        \t=int(floor((\note-1)/12)*7-7);
        \notename={"A","","B","C","","D","","E","F","","G",""}[\n];
        \tonicsolfa={"la","","si","so","","re","","mi","fa","","sol",""}[\n];
        \blacknote={0,1,0,0,1,0,1,0,0,1,0,1}[\n];
        \frequency=(2^((\note-49)/12))*440;}
] in {\minNote,...,\maxNote}{
    \ifnum\octave>3
        \tikzset{extract anchor/.style={anchor=south west, at=(\lastnotenodename.south east)}}
    \else
        \tikzset{extract anchor/.style={anchor=north west, at=(\lastnotenodename.north east)}}
    \fi
    \ifnum\blacknote=0
        \edef\notenodename{\notename_\octave}
        \node (\notenodename) [tight fit,text width=1cm, extract anchor/.try]  {%           
            \begin{music}
                %\instrumentnumber{1}
                %\instrumentnumber{2}
                \nostartrule        
                \setstaffs1{1}
                %\setstaffs2{1}  
                \setclefsymbol1{\empty}
                %\setclefsymbol2{\empty}     
                %\setclef1{\bass}
                \setclef1{\treble}                      
                \startextract
                \transpose\t
                \hskip-1.5\elemskip         
                %\ifnum\octave>3
                    %\ifnum\octave>4
                        %\Notes \nextinstrument \ql{\notename} \en       
                    %\else
                    %    \Notes \nextinstrument \qu{\notename} \en                       
                    %\fi
                %\else
                 %   \ifnum\octave>2
                        \Notes \ql{\notename} \en
                 %   \else
                 %       \Notes \qu{\notename} \en
                 %   \fi
                %\fi
                \zendextract
            \end{music}
        };
        \xdef\lastnotenodename{\notenodename}       
        %\node [anchor=base] (sol-fa)  at (\notenodename |- 0,-3) {\tonicsolfa$_\octave$};

        \draw (\notenodename.south west |- 0,-4) rectangle ++(1, -4);
        %\node [rotate=90, font=\footnotesize, anchor=east] 
         %   at (\notenodename.north |- 0,-4) {\threedp\frequency};
        %\node [font=\footnotesize, anchor=south]  
         %   at (\notenodename.south |- 0,-8) {\note};
        \node [font=\footnotesize, anchor=south] 
            at (\notenodename.south |- 0,-8.5)  {$\notename$};
        %%\draw (\notenodename.south west |- sol-fa.south) 
        %%    rectangle (\notenodename.south east |- 0,1.125); %0.125 by trial and error
    \else
    \ifnum\n=1 %% A#--shift right
            \begin{pgfonlayer}{blacknotes}
              \fill ([xshift=-0.2cm]\lastnotenodename.north east |- 0,-4) rectangle ++(0.6, -2.5);
              %\node  [rotate=90, text=white, font=\footnotesize, anchor=east]
              %    at (\lastnotenodename.north east |- 0,-4) {\threedp\frequency};
            \end{pgfonlayer}
            \fi
            \ifnum\n=4 %% C# -- shift left
            \begin{pgfonlayer}{blacknotes}
              \fill ([xshift=-0.4cm]\lastnotenodename.north east |- 0,-4) rectangle ++(0.6, -2.5);
              %\node  [rotate=90, text=white, font=\footnotesize, anchor=east]
              %    at (\lastnotenodename.north east |- 0,-4) {\threedp\frequency};
            \end{pgfonlayer}
            \fi
            \ifnum\n=6 %% D# -- shift right
            \begin{pgfonlayer}{blacknotes}
              \fill ([xshift=-0.2cm]\lastnotenodename.north east |- 0,-4) rectangle ++(0.6, -2.5);
              %\node  [rotate=90, text=white, font=\footnotesize, anchor=east]
              %    at (\lastnotenodename.north east |- 0,-4) {\threedp\frequency};
            \end{pgfonlayer}
            \fi
            \ifnum\n=9 %% F# -- shift left
            \begin{pgfonlayer}{blacknotes}
              \fill ([xshift=-0.4cm]\lastnotenodename.north east |- 0,-4) rectangle ++(0.6, -2.5);
              %\node  [rotate=90, text=white, font=\footnotesize, anchor=east]
              %    at (\lastnotenodename.north east |- 0,-4) {\threedp\frequency};
            \end{pgfonlayer}
            \fi
            \ifnum\n=11 %% G# -- Don't move
            \begin{pgfonlayer}{blacknotes}
              \fill ([xshift=-0.3cm]\lastnotenodename.north east |- 0,-4) rectangle ++(0.6, -2.5);
              %\node  [rotate=90, text=white, font=\footnotesize, anchor=east]
              %    at (\lastnotenodename.north east |- 0,-4) {\threedp\frequency};
            \end{pgfonlayer}
            \fi
            \fi
        %\node  [rotate=90, text=white, font=\footnotesize, anchor=east]
        %    at (\lastnotenodename.north east |- 0,-4) {\threedp\frequency};
       }
%\node [rotate=90] at (0,-6) {Frequency (Hz)};
\end{tikzpicture}
\end{center}%
%%%%%%%%%%%%%%%%%%%%%%%%%%%%%%%%%%%%%%%%%%%%%%%%%%%%%%%%%%%%%%%%%% 
%%%%%%%%%%%%%%%%%%%%%%%%%%%%%%%%%%%%%%%%%%%%%%%%%%%%%%%%%%%%%%%%%%  
}



\newcommand{\noStaff}{%% from: https://tex.stackexchange.com/questions/136713/range-of-a-piano-keyboard-vs-staves
%%%%%%%%%%%%%%%%%%%%%%%%%%%%%%%%%%%%%%%%%%%%%%%%%%%%%%%%%%%%%%%%%%
%%%%%%%%%%%%%%%%%%%%%%%%%%%%%%%%%%%%%%%%%%%%%%%%%%%%%%%%%%%%%%%%%%
%% 
\pgfdeclarelayer{blacknotes}
\pgfsetlayers{main,blacknotes}
\tikzset{tight fit/.style={inner sep=0pt, outer sep=0pt}}


\pgfdeclarelayer{blacknotes}
\pgfsetlayers{main,blacknotes}
\tikzset{tight fit/.style={inner sep=0pt, outer sep=0pt}}

\begin{center}
\vspace{-1.5in} %hack!
\begin{tikzpicture}[scale=1,every node/.style={transform shape}]
  \def\lastnotenodename{clefs}
  \def\minNote{37}
  \def\maxNote{61}
  \node [text width=1cm, tight fit] (clefs) at (0,0) {};
\foreach \note [
    evaluate={
        \n=int(mod(\note-1, 12));
        \octave=int((\note+8)/12);
        \t=int(floor((\note-1)/12)*7-7);
        \notename={"A","","B","C","","D","","E","F","","G",""}[\n];
        \tonicsolfa={"la","","si","so","","re","","mi","fa","","sol",""}[\n];
        \blacknote={0,1,0,0,1,0,1,0,0,1,0,1}[\n];
        \frequency=(2^((\note-49)/12))*440;}
] in {\minNote,...,\maxNote}{

    \ifnum\octave>3
        \tikzset{extract anchor/.style={anchor=south west, at=(\lastnotenodename.south east)}}
    \else
        \tikzset{extract anchor/.style={anchor=north west, at=(\lastnotenodename.north east)}}
    \fi
    \ifnum\blacknote=0
        \edef\notenodename{\notename_\octave}
        \node (\notenodename) [tight fit,text width=1cm, extract anchor/.try]  {};
        \xdef\lastnotenodename{\notenodename}       
        %\node [anchor=base] (sol-fa)  at (\notenodename |- 0,-3) {\tonicsolfa$_\octave$};

        \draw (\notenodename.south west |- 0,-4) rectangle ++(1, -4);
        \ifnum\n=0
        \node [xshift=0cm,rotate=90, font=\footnotesize, anchor=east] 
        at (\notenodename.north |- 0,-4) {\threedp\frequency};
        \fi
         \ifnum\n=2
        \node [xshift=0.2cm,rotate=90, font=\footnotesize, anchor=east] 
        at (\notenodename.north |- 0,-4) {\threedp\frequency};
        \fi
        \ifnum\n=3
        \node [xshift=-0.2cm,rotate=90, font=\footnotesize, anchor=east] 
        at (\notenodename.north |- 0,-4) {\threedp\frequency};
        \fi
        \ifnum\n=5
        \node [rotate=90, font=\footnotesize, anchor=east] 
        at (\notenodename.north |- 0,-4) {\threedp\frequency};
        \fi
        \ifnum\n=7
        \node [xshift=0.2cm,rotate=90, font=\footnotesize, anchor=east] 
        at (\notenodename.north |- 0,-4) {\threedp\frequency};
        \fi
        \ifnum\n=8
        \node [xshift=-0.2cm,rotate=90, font=\footnotesize, anchor=east] 
        at (\notenodename.north |- 0,-4) {\threedp\frequency};
        \fi
        \ifnum\n=10
        \node [xshift=0.0cm,rotate=90, font=\footnotesize, anchor=east] 
        at (\notenodename.north |- 0,-4) {\threedp\frequency};
        \fi

        \node [font=\footnotesize, anchor=south]  
            at (\notenodename.south |- 0,-8) {\note};
        \node [font=\footnotesize, anchor=south] 
            at (\notenodename.south |- 0,-8.5)  {$\notename_\octave$};
            \else
            \ifnum\n=1 %% A#--shift right
            \begin{pgfonlayer}{blacknotes}
              \fill ([xshift=-0.2cm]\lastnotenodename.north east |- 0,-4) rectangle ++(0.6, -2.5);
              \node  [xshift=0.1cm,rotate=90, text=white, font=\footnotesize, anchor=east]
                  at (\lastnotenodename.north east |- 0,-4) {\threedp\frequency};
            \end{pgfonlayer}
            \fi
            \ifnum\n=4 %% C# -- shift left
            \begin{pgfonlayer}{blacknotes}
              \fill ([xshift=-0.4cm]\lastnotenodename.north east |- 0,-4) rectangle ++(0.6, -2.5);
              \node  [xshift=-0.1cm,rotate=90, text=white, font=\footnotesize, anchor=east]
                  at (\lastnotenodename.north east |- 0,-4) {\threedp\frequency};
            \end{pgfonlayer}
            \fi
            \ifnum\n=6 %% D# -- shift right
            \begin{pgfonlayer}{blacknotes}
              \fill ([xshift=-0.2cm]\lastnotenodename.north east |- 0,-4) rectangle ++(0.6, -2.5);
              \node  [xshift=0.1cm,rotate=90, text=white, font=\footnotesize, anchor=east]
                  at (\lastnotenodename.north east |- 0,-4) {\threedp\frequency};
            \end{pgfonlayer}
            \fi
            \ifnum\n=9 %% F# -- shift left
            \begin{pgfonlayer}{blacknotes}
              \fill ([xshift=-0.4cm]\lastnotenodename.north east |- 0,-4) rectangle ++(0.6, -2.5);
              \node  [xshift=-0.1cm,rotate=90, text=white, font=\footnotesize, anchor=east]
                  at (\lastnotenodename.north east |- 0,-4) {\threedp\frequency};
            \end{pgfonlayer}
            \fi
            \ifnum\n=11 %% G# -- Don't move
            \begin{pgfonlayer}{blacknotes}
              \fill ([xshift=-0.3cm]\lastnotenodename.north east |- 0,-4) rectangle ++(0.6, -2.5);
              \node  [rotate=90, text=white, font=\footnotesize, anchor=east]
                  at (\lastnotenodename.north east |- 0,-4) {\threedp\frequency};
            \end{pgfonlayer}
            \fi
            \fi
            }
\node [rotate=90] at (0,-6) {Frequency (Hz)};
\end{tikzpicture}
\end{center}%
%%%%%%%%%%%%%%%%%%%%%%%%%%%%%%%%%%%%%%%%%%%%%%%%%%%%%%%%%%%%%%%%%% 
%%%%%%%%%%%%%%%%%%%%%%%%%%%%%%%%%%%%%%%%%%%%%%%%%%%%%%%%%%%%%%%%%%  
}


\newcommand{\staffBack}{%% from: https://tex.stackexchange.com/questions/136713/range-of-a-piano-keyboard-vs-staves
%%%%%%%%%%%%%%%%%%%%%%%%%%%%%%%%%%%%%%%%%%%%%%%%%%%%%%%%%%%%%%%%%%
%%%%%%%%%%%%%%%%%%%%%%%%%%%%%%%%%%%%%%%%%%%%%%%%%%%%%%%%%%%%%%%%%%
%% 
\pgfdeclarelayer{blacknotes}
\pgfsetlayers{main,blacknotes}
\tikzset{tight fit/.style={inner sep=0pt, outer sep=0pt}}


\pgfdeclarelayer{blacknotes}
\pgfsetlayers{main,blacknotes}
\tikzset{tight fit/.style={inner sep=0pt, outer sep=0pt}}

\begin{center}
\begin{tikzpicture}[scale=1,every node/.style={transform shape}]
  \def\lastnotenodename{clefs}
  \def\minNote{40}
  \def\maxNote{61}
\node [text width=1cm, tight fit] (clefs) at (0,-3) {
    \begin{music}
        \instrumentnumber{1}
        %\instrumentnumber{2}
        \nostartrule        
        \setstaffs1{1}
        %\setstaffs2{1}  
        %\setclef1{\bass}
        \setclef1{\treble}                                  
        \startextract
        \hskip2.0\elemskip
        \zendextract
    \end{music}
};

\foreach \note [
    evaluate={
        \n=int(mod(\note-1, 12));
        \octave=int((\note+8)/12);
        \t=int(floor((\note-1)/12)*7-7);
        \notename={"A","","B","C","","D","","E","F","","G",""}[\n];
        \tonicsolfa={"la","","si","so","","re","","mi","fa","","sol",""}[\n];
        \blacknote={0,1,0,0,1,0,1,0,0,1,0,1}[\n];
        \frequency=(2^((\note-49)/12))*440;}
] in {\minNote,...,\maxNote}{
    \ifnum\octave>3
        \tikzset{extract anchor/.style={anchor=south west, at=(\lastnotenodename.south east)}}
    \else
        \tikzset{extract anchor/.style={anchor=north west, at=(\lastnotenodename.north east)}}
    \fi
    \ifnum\blacknote=0
        \edef\notenodename{\notename_\octave}
        \node (\notenodename) [tight fit,text width=1cm, extract anchor/.try]  {%           
            \begin{music}
                %\instrumentnumber{1}
                %\instrumentnumber{2}
                \nostartrule        
                \setstaffs1{1}
                %\setstaffs2{1}  
                \setclefsymbol1{\empty}
                %\setclefsymbol2{\empty}     
                %\setclef1{\bass}
                \setclef1{\treble}                      
                \startextract
                \transpose\t
                \hskip-1.5\elemskip         
                \ifnum\octave>3
                    \ifnum\octave>4
                \Notes  \ql{\notename} \en       
                    \else
                       \Notes \qu{\notename} \en                       
                       \fi
                       \fi
                %\else
                   % \ifnum\octave>2
                    %    \Notes \ql{\notename} \en
                    %\else
                     %   \Notes \qu{\notename} \en
                    %\fi
                %\fi
                \zendextract
            \end{music}
        };
        \xdef\lastnotenodename{\notenodename}       
        %\node [anchor=base] (sol-fa)  at (\notenodename |- 0,-3) {\tonicsolfa$_\octave$};

        \draw (\notenodename.south west |- 0,-4) rectangle ++(1, -4);
         \ifnum\n=0
        \node [xshift=0cm,rotate=90, font=\footnotesize, anchor=east] 
        at (\notenodename.north |- 0,-4) {\threedp\frequency};
        \fi
        \ifnum\n=2
        \node [xshift=0.2cm,rotate=90, font=\footnotesize, anchor=east] 
        at (\notenodename.north |- 0,-4) {\threedp\frequency};
        \fi
        \ifnum\n=3
        \node [xshift=-0.2cm,rotate=90, font=\footnotesize, anchor=east] 
        at (\notenodename.north |- 0,-4) {\threedp\frequency};
        \fi
        \ifnum\n=5
        \node [rotate=90, font=\footnotesize, anchor=east] 
        at (\notenodename.north |- 0,-4) {\threedp\frequency};
        \fi
        \ifnum\n=7
        \node [xshift=0.2cm,rotate=90, font=\footnotesize, anchor=east] 
        at (\notenodename.north |- 0,-4) {\threedp\frequency};
        \fi
        \ifnum\n=8
        \node [xshift=-0.2cm,rotate=90, font=\footnotesize, anchor=east] 
        at (\notenodename.north |- 0,-4) {\threedp\frequency};
        \fi
        \ifnum\n=10
        \node [xshift=0.0cm,rotate=90, font=\footnotesize, anchor=east] 
        at (\notenodename.north |- 0,-4) {\threedp\frequency};
        \fi
        %\node [font=\footnotesize, anchor=south]  
         %   at (\notenodename.south |- 0,-8) {\note};
        \node [font=\footnotesize, anchor=south] 
            at (\notenodename.south |- 0,-8.5)  {$\notename_\octave$};
        %%\draw (\notenodename.south west |- sol-fa.south) 
        %%    rectangle (\notenodename.south east |- 0,1.125); %0.125 by trial and error
            \else
             \ifnum\n=1 %% A#--shift right
            \begin{pgfonlayer}{blacknotes}
              \fill ([xshift=-0.2cm]\lastnotenodename.north east |- 0,-4) rectangle ++(0.6, -2.5);
              \node  [xshift=0.1cm,rotate=90, text=white, font=\footnotesize, anchor=east]
                  at (\lastnotenodename.north east |- 0,-4) {\threedp\frequency};
            \end{pgfonlayer}
            \fi
            \ifnum\n=4 %% C# -- shift left
            \begin{pgfonlayer}{blacknotes}
              \fill ([xshift=-0.4cm]\lastnotenodename.north east |- 0,-4) rectangle ++(0.6, -2.5);
              \node  [xshift=-0.1cm,rotate=90, text=white, font=\footnotesize, anchor=east]
                  at (\lastnotenodename.north east |- 0,-4) {\threedp\frequency};
            \end{pgfonlayer}
            \fi
            \ifnum\n=6 %% D# -- shift right
            \begin{pgfonlayer}{blacknotes}
              \fill ([xshift=-0.2cm]\lastnotenodename.north east |- 0,-4) rectangle ++(0.6, -2.5);
              \node  [xshift=0.1cm,rotate=90, text=white, font=\footnotesize, anchor=east]
                  at (\lastnotenodename.north east |- 0,-4) {\threedp\frequency};
            \end{pgfonlayer}
            \fi
            \ifnum\n=9 %% F# -- shift left
            \begin{pgfonlayer}{blacknotes}
              \fill ([xshift=-0.4cm]\lastnotenodename.north east |- 0,-4) rectangle ++(0.6, -2.5);
              \node  [xshift=-0.1cm,rotate=90, text=white, font=\footnotesize, anchor=east]
                  at (\lastnotenodename.north east |- 0,-4) {\threedp\frequency};
            \end{pgfonlayer}
            \fi
            \ifnum\n=11 %% G# -- Don't move
            \begin{pgfonlayer}{blacknotes}
              \fill ([xshift=-0.3cm]\lastnotenodename.north east |- 0,-4) rectangle ++(0.6, -2.5);
              \node  [rotate=90, text=white, font=\footnotesize, anchor=east]
                  at (\lastnotenodename.north east |- 0,-4) {\threedp\frequency};
            \end{pgfonlayer}
            \fi
            \fi
}
\node [rotate=90] at (0,-6) {Frequency (Hz)};
\end{tikzpicture}
\end{center}%
%%%%%%%%%%%%%%%%%%%%%%%%%%%%%%%%%%%%%%%%%%%%%%%%%%%%%%%%%%%%%%%%%% 
%%%%%%%%%%%%%%%%%%%%%%%%%%%%%%%%%%%%%%%%%%%%%%%%%%%%%%%%%%%%%%%%%%  
}








\newcommand{\sharpStaff}{
\pgfdeclarelayer{blacknotes}
\pgfsetlayers{main,blacknotes}
\tikzset{tight fit/.style={inner sep=0pt, outer sep=0pt}}
%\def\t{14}
\def\ms{\hspace{4.5pt}}
\begin{center}
\begin{tikzpicture}[scale=1,every node/.style={transform shape}]
 \def\lastnotenodename{clefs}
  \def\minNote{40}
  \def\maxNote{51}
\node [text width=1cm, tight fit] (clefs) at (0,-3) {};
  \node[tight fit,anchor=west]  at (-5,-3){%           
            \begin{music}
                %\instrumentnumber{1}
                %\instrumentnumber{2}
                \nostartrule        
                \setstaffs1{1}
                %\setstaffs2{1}  
                %\setclefsymbol1{\empty}
                %\setclefsymbol2{\empty}     
                %\setclef1{\bass}
                \setclef1{\treble}                      
                \startextract
                %\transpose14
                %\hskip-1.5\elemskip         
                %\ifnum\octave>3
                    %\ifnum\octave>4
                        %\Notes \nextinstrument \ql{\notename} \en       
                    %\else
                    %    \Notes \nextinstrument \qu{\notename} \en                       
                    %\fi
                %\else
                    %\ifnum\octave>4
                     %   \Notes \ql{\notename}   \ql{\sh\notename}                 \en
                    %\else
                \notes\phantom{\qu{c}}\qu{c}\ms\qu{^c}\ms\qu{d}\ms\qu{^d}\ms\qu{e}\ms\qu{f}\ms\qu{^f}\ms\qu{g}\ms\qu{^g}\ms\qu{'a}\ms\qu{^a}\ms\qu{b}\en
                    %\fi
                %\fi
                \zendextract
            \end{music}
        };

\foreach \note [
    evaluate={
        \n=int(mod(\note-1, 12));
        \octave=int((\note+8)/12);
        \t=int(floor((\note-1)/12)*7-7);
        \notename={"A","","B","C","","D","","E","F","","G",""}[\n];
        \tonicsolfa={"la","","si","so","","re","","mi","fa","","sol",""}[\n];
        \blacknote={0,1,0,0,1,0,1,0,0,1,0,1}[\n];
        \frequency=(2^((\note-49)/12))*440;}
] in {\minNote,...,\maxNote}{
    \ifnum\octave>3
        \tikzset{extract anchor/.style={anchor=south west, at=(\lastnotenodename.south east)}}
    \else
        \tikzset{extract anchor/.style={anchor=north west, at=(\lastnotenodename.north east)}}
    \fi
    \ifnum\blacknote=0
    \edef\notenodename{\notename_\octave}
    \xdef\lastnotenodename{\notenodename}       
        %\node [anchor=base] (sol-fa)  at (\notenodename |- 0,-3) {\tonicsolfa$_\octave$};

        \draw (\notenodename.south west |- 0,-4) rectangle ++(1, -4);
        %\node [rotate=90, font=\footnotesize, anchor=east] 
         %   at (\notenodename.north |- 0,-4) {\threedp\frequency};
        %\node [font=\footnotesize, anchor=south]  
         %   at (\notenodename.south |- 0,-8) {\note};
        \node [font=\footnotesize, anchor=south] 
            at (\notenodename.south |- 0,-8.5)  {$\notename$};
        %%\draw (\notenodename.south west |- sol-fa.south) 
        %%    rectangle (\notenodename.south east |- 0,1.125); %0.125 by trial and error
            \else
            \ifnum\n=1 %% A#--shift right
            \begin{pgfonlayer}{blacknotes}
              \fill ([xshift=-0.2cm]\lastnotenodename.north east |- 0,-4) rectangle ++(0.6, -2.5);
              %\node  [rotate=90, text=white, font=\footnotesize, anchor=east]
              %    at (\lastnotenodename.north east |- 0,-4) {\threedp\frequency};
            \end{pgfonlayer}
            \fi
            \ifnum\n=4 %% C# -- shift left
            \begin{pgfonlayer}{blacknotes}
              \fill ([xshift=-0.4cm]\lastnotenodename.north east |- 0,-4) rectangle ++(0.6, -2.5);
              %\node  [rotate=90, text=white, font=\footnotesize, anchor=east]
              %    at (\lastnotenodename.north east |- 0,-4) {\threedp\frequency};
            \end{pgfonlayer}
            \fi
            \ifnum\n=6 %% D# -- shift right
            \begin{pgfonlayer}{blacknotes}
              \fill ([xshift=-0.2cm]\lastnotenodename.north east |- 0,-4) rectangle ++(0.6, -2.5);
              %\node  [rotate=90, text=white, font=\footnotesize, anchor=east]
              %    at (\lastnotenodename.north east |- 0,-4) {\threedp\frequency};
            \end{pgfonlayer}
            \fi
            \ifnum\n=9 %% F# -- shift left
            \begin{pgfonlayer}{blacknotes}
              \fill ([xshift=-0.4cm]\lastnotenodename.north east |- 0,-4) rectangle ++(0.6, -2.5);
              %\node  [rotate=90, text=white, font=\footnotesize, anchor=east]
              %    at (\lastnotenodename.north east |- 0,-4) {\threedp\frequency};
            \end{pgfonlayer}
            \fi
            \ifnum\n=11 %% G# -- Don't move
            \begin{pgfonlayer}{blacknotes}
              \fill ([xshift=-0.3cm]\lastnotenodename.north east |- 0,-4) rectangle ++(0.6, -2.5);
              %\node  [rotate=90, text=white, font=\footnotesize, anchor=east]
              %    at (\lastnotenodename.north east |- 0,-4) {\threedp\frequency};
            \end{pgfonlayer}
            \fi
            \fi
}
%\node [rotate=90] at (0,-6) {Frequency (Hz)};
\end{tikzpicture}
\end{center}%
}


\newcommand{\ttPianoNoStaff}{%% from: https://tex.stackexchange.com/questions/136713/range-of-a-piano-keyboard-vs-staves
%%%%%%%%%%%%%%%%%%%%%%%%%%%%%%%%%%%%%%%%%%%%%%%%%%%%%%%%%%%%%%%%%%
%%%%%%%%%%%%%%%%%%%%%%%%%%%%%%%%%%%%%%%%%%%%%%%%%%%%%%%%%%%%%%%%%%
%% 
\pgfdeclarelayer{blacknotes}
\pgfsetlayers{main,blacknotes}
\tikzset{tight fit/.style={inner sep=0pt, outer sep=0pt}}

\begin{center}
\begin{tikzpicture}[scale=1,every node/.style={transform shape}]
  \def\lastnotenodename{clefs}
  \def\minNote{40}
  \def\maxNote{51}
\node [text width=1cm, tight fit] (clefs) at (0,-3) {};
  \node[above] at (1.4,-4) {$C\sharp$};
  \node[above] at (2.6,-4) {$D\sharp$};
  \node[above] at (4.4,-4) {$F\sharp$};
  \node[above] at (5.6,-4) {$G\sharp$};
  \node[above] at (6.6,-4) {$A\sharp$};
\foreach \note [
    evaluate={
        \n=int(mod(\note-1, 12));
        \octave=int((\note+8)/12);
        \t=int(floor((\note-1)/12)*7-7);
        \notename={"A","","B","C","","D","","E","F","","G",""}[\n];
        \tonicsolfa={"la","","si","so","","re","","mi","fa","","sol",""}[\n];
        \blacknote={0,1,0,0,1,0,1,0,0,1,0,1}[\n];
        \frequency=(2^((\note-49)/12))*440;}
] in {\minNote,...,\maxNote}{
    \ifnum\octave>3
        \tikzset{extract anchor/.style={anchor=south west, at=(\lastnotenodename.south east)}}
    \else
        \tikzset{extract anchor/.style={anchor=north west, at=(\lastnotenodename.north east)}}
    \fi
    \ifnum\blacknote=0
        \edef\notenodename{\notename_\octave}
        \node (\notenodename) [tight fit,text width=1cm, extract anchor/.try]{};
        \xdef\lastnotenodename{\notenodename}       
        %\node [anchor=base] (sol-fa)  at (\notenodename |- 0,-3) {\tonicsolfa$_\octave$};

        \draw (\notenodename.south west |- 0,-4) rectangle ++(1, -4);
        %\node [rotate=90, font=\footnotesize, anchor=east] 
         %   at (\notenodename.north |- 0,-4) {\threedp\frequency};
        %\node [font=\footnotesize, anchor=south]  
         %   at (\notenodename.south |- 0,-8) {\note};
        \node [font=\footnotesize, anchor=south] 
            at (\notenodename.south |- 0,-8.5)  {$\notename$};
        %%\draw (\notenodename.south west |- sol-fa.south) 
        %%    rectangle (\notenodename.south east |- 0,1.125); %0.125 by trial and error
    \else
    \ifnum\n=1 %% A#--shift right
            \begin{pgfonlayer}{blacknotes}
              \fill ([xshift=-0.2cm]\lastnotenodename.north east |- 0,-4) rectangle ++(0.6, -2.5);
              %\node  [rotate=90, text=white, font=\footnotesize, anchor=east]
              %    at (\lastnotenodename.north east |- 0,-4) {\threedp\frequency};
            \end{pgfonlayer}
            \fi
            \ifnum\n=4 %% C# -- shift left
            \begin{pgfonlayer}{blacknotes}
              \fill ([xshift=-0.4cm]\lastnotenodename.north east |- 0,-4) rectangle ++(0.6, -2.5);
              %\node  [rotate=90, text=white, font=\footnotesize, anchor=east]
              %    at (\lastnotenodename.north east |- 0,-4) {\threedp\frequency};
            \end{pgfonlayer}
            \fi
            \ifnum\n=6 %% D# -- shift right
            \begin{pgfonlayer}{blacknotes}
              \fill ([xshift=-0.2cm]\lastnotenodename.north east |- 0,-4) rectangle ++(0.6, -2.5);
              %\node  [rotate=90, text=white, font=\footnotesize, anchor=east]
              %    at (\lastnotenodename.north east |- 0,-4) {\threedp\frequency};
            \end{pgfonlayer}
            \fi
            \ifnum\n=9 %% F# -- shift left
            \begin{pgfonlayer}{blacknotes}
              \fill ([xshift=-0.4cm]\lastnotenodename.north east |- 0,-4) rectangle ++(0.6, -2.5);
              %\node  [rotate=90, text=white, font=\footnotesize, anchor=east]
              %    at (\lastnotenodename.north east |- 0,-4) {\threedp\frequency};
            \end{pgfonlayer}
            \fi
            \ifnum\n=11 %% G# -- Don't move
            \begin{pgfonlayer}{blacknotes}
              \fill ([xshift=-0.3cm]\lastnotenodename.north east |- 0,-4) rectangle ++(0.6, -2.5);
              %\node  [rotate=90, text=white, font=\footnotesize, anchor=east]
              %    at (\lastnotenodename.north east |- 0,-4) {\threedp\frequency};
            \end{pgfonlayer}
            \fi
            \fi
        %\node  [rotate=90, text=white, font=\footnotesize, anchor=east]
        %    at (\lastnotenodename.north east |- 0,-4) {\threedp\frequency};
       }
%\node [rotate=90] at (0,-6) {Frequency (Hz)};
\end{tikzpicture}
\end{center}%
%%%%%%%%%%%%%%%%%%%%%%%%%%%%%%%%%%%%%%%%%%%%%%%%%%%%%%%%%%%%%%%%%% 
%%%%%%%%%%%%%%%%%%%%%%%%%%%%%%%%%%%%%%%%%%%%%%%%%%%%%%%%%%%%%%%%%%  
}




%% Circle tours
\usetikzlibrary{bending,positioning}
\usetikzlibrary{calc}
\usepackage{xstring}

%https://tex.stackexchange.com/questions/21559/macro-to-access-a-specific-member-of-a-list
% This works both with inline lists and with macros containing lists
\newcommand*{\GetListMember}[2]{%
    \edef\dotheloop{%
    \noexpand\foreach \noexpand\a [count=\noexpand\i] in {#1} {%
        \noexpand\IfEq{\noexpand\i}{#2}{\noexpand\a\noexpand\breakforeach}{}%
    }}%
    \dotheloop
    \par%
}%
%%%% 
%%%% Circle tours
%%%%
% code from Heiko Oberdiek
% https://tex.stackexchange.com/a/250270/3954
\newcommand{\circleTour}[2]{%
\begin{tikzpicture}[
  ->,   
  thick,
  main node/.style={},
]
  \newcommand*{\MainNum}{12}
  \newcommand*{\MainRadius}{3cm} 
  \newcommand*{\MainStartAngle}{90}
  % Print main nodes, node names: p1, p2, ...
  \path
    (0, 0) coordinate (M)
    \foreach \t [count=\i] in {1,...,12} {
      +({1-\i)*360/\MainNum + \MainStartAngle}:\MainRadius)
      node[main node, align=center] (species_\i) {\GetListMember{#1}{\i}}
    }
  ;  

  % Calculate the angle between the equal sides of the triangle
  % with side length \MainRadius, \MainRadius and radius of circle node
  % Result is stored in \p1-angle, \p2-angle, ...
  \foreach \i in {1, ..., \MainNum} {
    \pgfextracty{\dimen0 }{\pgfpointanchor{species_\i}{north}} 
    \pgfextracty{\dimen2 }{\pgfpointanchor{species_\i}{center}}
    \dimen0=\dimexpr\dimen2 - \dimen0\relax 
    \ifdim\dimen0<0pt \dimen0 = -\dimen0 \fi
    \pgfmathparse{2*asin(\the\dimen0/\MainRadius/2)}
    \global\expandafter\let\csname p\i-angle\endcsname\pgfmathresult
  }

  % Draw the arrow arcs
  \foreach \i [evaluate=\i as \nexti using {int(mod(\i, \MainNum)+1}]
  in {1, ..., \MainNum} {  
    \pgfmathsetmacro\StartAngle{   
      (\i-1)*360/\MainNum + \MainStartAngle
      + \csname p\i-angle\endcsname
    }
    \pgfmathsetmacro\EndAngle{
      (\nexti-1)*360/\MainNum + \MainStartAngle
      - \csname p\nexti-angle\endcsname
    }
    \ifdim\EndAngle pt < \StartAngle pt
      \pgfmathsetmacro\EndAngle{\EndAngle + 360}
    \fi
    \draw[<-]
      (M) ++(\StartAngle:\MainRadius)
      arc[start angle=\StartAngle, end angle=\EndAngle, radius=\MainRadius]
       node[midway,above,sloped] {\tiny{$#2$}};    
    ;
  }
\end{tikzpicture}}
%%END CIRCLE TOURS






%%%%%%%%%%%%%%%%%
%%% SCALE CIRCLES
%%%%%%%%%%%%%%%%%

%%%% 
%%%% Circle tours
%%%%
% code from Heiko Oberdiek
% https://tex.stackexchange.com/a/250270/3954
\newcommand{\circlescale}[2]{%
\begin{tikzpicture}[
  ->,   
  thick,
  main node/.style={},
]
  \newcommand*{\MainNum}{12}
  \newcommand*{\MainRadius}{3cm} 
  \newcommand*{\MainStartAngle}{90}
  % Print main nodes, node names: p1, p2, ...
  \path
    (0, 0) coordinate (M)
    \foreach \t [count=\i] in {1,...,12} {
      +({1-\i)*360/\MainNum + \MainStartAngle}:\MainRadius)
      node[main node, align=center] (species_\i) {\GetListMember{#1}{\i}}
    }
  ;  

  % Calculate the angle between the equal sides of the triangle
  % with side length \MainRadius, \MainRadius and radius of circle node
  % Result is stored in \p1-angle, \p2-angle, ...
  \foreach \i in {1, ..., \MainNum} {
    \pgfextracty{\dimen0 }{\pgfpointanchor{species_\i}{north}} 
    \pgfextracty{\dimen2 }{\pgfpointanchor{species_\i}{center}}
    \dimen0=\dimexpr\dimen2 - \dimen0\relax 
    \ifdim\dimen0<0pt \dimen0 = -\dimen0 \fi
    \pgfmathparse{2*asin(\the\dimen0/\MainRadius/2)}
    \global\expandafter\let\csname p\i-angle\endcsname\pgfmathresult
  }

  % Draw the arrow arcs
  \foreach \i [evaluate=\i as \nexti using {int(mod(\i, \MainNum)+1}]
  in {1, ..., \MainNum} {  
    \pgfmathsetmacro\StartAngle{   
      (\i-1)*360/\MainNum + \MainStartAngle
      + \csname p\i-angle\endcsname
    }
    \pgfmathsetmacro\EndAngle{
      (\nexti-1)*360/\MainNum + \MainStartAngle
      - \csname p\nexti-angle\endcsname
    }
    \ifdim\EndAngle pt < \StartAngle pt
      \pgfmathsetmacro\EndAngle{\EndAngle + 360}
    \fi
    \draw[<-]
      (M) ++(\StartAngle:\MainRadius)
      arc[start angle=\StartAngle, end angle=\EndAngle, radius=\MainRadius]
       node[midway,above,sloped] {\tiny{$#2$}};    
    ;
  }
\end{tikzpicture}}


%%%%%%
%% FROM: https://tikz.net/waves_standing/
%%%%%%
\colorlet{metalcol}{black!30!white}
\tikzstyle{vvec}=[->,vcol,very thick,line cap=round]
\tikzstyle{node}=[xcol,scale=0.8]
\tikzstyle{metal}=[draw=metalcol!10!black,rounded corners=0.1,
  top color=metalcol,bottom color=metalcol!80!black,shading angle=10]
\tikzstyle{ring}=[metalcol!20!black,double=metalcol!70!black,double distance=1.2,line width=0.3]
\tikzstyle{rope}=[white!10!black,double=white!50!black,
  double distance=1.2,line width=0.6] %very thick,line cap=round
\tikzstyle{wood}=[draw=white!80!black,rounded corners=0.1,
  black!50!white]
\def\tick#1#2{\draw[thick] (#1) ++ (#2:0.1) --++ (#2-180:0.2)}
\tikzstyle{myarr}=[-{Latex[length=3,width=2]},vcol!40]
\tikzstyle{mydoublearr}=[{Latex[length=3,width=2]}-{Latex[length=3,width=2]},vcol!40]


% STANDING WAVE - DOUBLE FIXED - n=1
\def\L{5.0}
\def\A{0.6}
\def\t{0.12}
\def\om{360/(\lam)} % omega (degrees)
\def\wave#1{
  \def\lam{2*\L/#1} % wavelength
  \draw[wood] (0,-1.4*\A) rectangle++ (-\t,2.8*\A);
  \draw[wood] (\L,-1.4*\A) rectangle++ ( \t,2.8*\A);
  \draw[dashed,samples=100,smooth,variable=\x,domain=0:\L]
   plot(\x,{-\A*sin(\om*\x)});
  %\draw[rope,samples=100,smooth,variable=\x,domain=0:\L]
   % plot(\x,{-\A*sin(\om*\x)});
  \draw[rope,samples=100,smooth,variable=\x,domain=0:\L]
    plot(\x,{\A*sin(\om*\x)});
  \draw[metal] (0,0) circle(0.7*\t);
  \draw[metal] (\L,0) circle(0.7*\t);
}
%% END STANDING Wave



% Transverse WAVE - DOUBLE FIXED - n=5 but p = ?
\def\L{5.0}
\def\A{0.6}
\def\t{0.12}
\def\n{5}
\def\om{360/(\lam)} % omega (degrees)
\def\transverseWave#1{
  \def\lam{2*\L/\n} % wavelength
  \draw[wood] (0,-1.4*\A) rectangle++ (-\t,2.8*\A);
  \draw[wood] (\L,-1.4*\A) rectangle++ ( \t,2.8*\A);
  %\draw[dashed,samples=100,smooth,variable=\x,domain=2:3]
   % plot(\x,{-\A*sin(\om*\x)});
  \draw[rope,samples=100,smooth,variable=\x,domain=#1:#1+1]
    plot(\x,{\A*sin(\om*(\x-#1))});

  \draw[rope,variable=\x,domain=0:#1]
    plot(\x,0);
  \draw[rope,samples=100,smooth,variable=\x,domain=#1+1:5]
    plot(\x,0);
  \draw[metal] (0,0) circle(0.7*\t);
  \draw[metal] (\L,0) circle(0.7*\t);
}




% Return Transverse WAVE - DOUBLE FIXED - n=5 but p = ?
\def\L{5.0}
\def\A{0.6}
\def\t{0.12}
\def\n{5}
\def\om{360/(\lam)} % omega (degrees)
\def\retTransverseWave#1{
  \def\lam{2*\L/\n} % wavelength
  \draw[wood] (0,-1.4*\A) rectangle++ (-\t,2.8*\A);
  \draw[wood] (\L,-1.4*\A) rectangle++ ( \t,2.8*\A);
  %\draw[dashed,samples=100,smooth,variable=\x,domain=2:3]
   % plot(\x,{-\A*sin(\om*\x)});
  \draw[rope,samples=100,smooth,variable=\x,domain=#1:#1+1]
    plot(\x,{-\A*sin(\om*(\x-#1))});

  \draw[rope,variable=\x,domain=0:#1]
    plot(\x,0);
  \draw[rope,samples=100,smooth,variable=\x,domain=#1+1:5]
    plot(\x,0);
  \draw[metal] (0,0) circle(0.7*\t);
  \draw[metal] (\L,0) circle(0.7*\t);
}

%% END TRANSVERSE Wave


%%MORE WAVES






% COLLIDE WAVE
\def\L{5.0}
\def\A{0.6}
\def\t{0.12}
\def\n{5}
\def\om{360/(\lam)} % omega (degrees)
\def\collideWave#1{
  \def\lam{2*\L/\n} % wavelength
  \draw[wood] (0,-1.4*\A) rectangle++ (-\t,2.8*\A);
  \draw[wood] (\L,-1.4*\A) rectangle++ ( \t,2.8*\A);
  %\draw[dashed,samples=100,smooth,variable=\x,domain=2:3]
   % plot(\x,{-\A*sin(\om*\x)});
  \draw[rope,samples=100,smooth,variable=\x,domain=#1:#1+1]
  plot(\x,{\A*sin(\om*(\x-#1))});
  \draw[rope,samples=100,smooth,variable=\x,domain=5-(#1+1):5-#1]
    plot(\x,{\A*sin(\om*((5-#1)-\x))});

  \draw[rope,variable=\x,domain=0:#1]
    plot(\x,0);
  \draw[rope,samples=100,smooth,variable=\x,domain=#1+1:5-#1-1]
  plot(\x,0);
  \draw[rope,samples=100,smooth,variable=\x,domain=5-#1:5]
    plot(\x,0);
  \draw[metal] (0,0) circle(0.7*\t);
  \draw[metal] (\L,0) circle(0.7*\t);
}


% Big WAVE
\def\n{5}
\def\om{360/(\lam)} % omega (degrees)
\def\bigWave#1{
  \def\lam{2*\L/\n} % wavelength
  \draw[wood] (0,-1.4*\A) rectangle++ (-\t,2.8*\A);
  \draw[wood] (\L,-1.4*\A) rectangle++ ( \t,2.8*\A);
  %\draw[dashed,samples=100,smooth,variable=\x,domain=2:3]
   % plot(\x,{-\A*sin(\om*\x)});
  \draw[rope,samples=100,smooth,variable=\x,domain=2:2+1]
  plot(\x,{1.2*sin(\om*(\x-2))});
  
  \draw[rope,variable=\x,domain=0:2]
    plot(\x,0);
  \draw[rope,samples=100,smooth,variable=\x,domain=3:5]
  plot(\x,0);
  \draw[metal] (0,0) circle(0.7*\t);
  \draw[metal] (\L,0) circle(0.7*\t);
}













































\usepackage{framed}
\usepackage{comment}

\let\instructorNotes\relax
\let\endinstructorNotes\relax
\let\instructorIntro\relax
\let\endinstructorIntro\relax

\ifinstructornotes
\newenvironment{instructorNotes}{\begin{framed}\noindent \textbf{Instructor note:}}{\end{framed}}
\newenvironment{instructorIntro}{\begin{framed}\noindent \textbf{Instructor introduction:}}{\end{framed}}
\newcommand{\teachingnotes}{With Teaching Notes}
\else
        \newenvironment{instructorNotes}%
                       {%
                         \setbox0\vbox\bgroup
                       }
{\egroup\ignorespacesafterend
                       }
        \newenvironment{instructorIntro}%
                       {%
                         \setbox0\vbox\bgroup
                       }
{\egroup\ignorespacesafterend
                       }
\newcommand{\teachingnotes}{}                       
                       \fi      
                       
                       
                       
                       
                       
                       
                       
                       
%%%% 
%%%% Circle tours
%%%%
% code from Heiko Oberdiek
% https://tex.stackexchange.com/a/250270/3954
\newcommand{\circlenumb}[2]{%
\begin{tikzpicture}[
  ->,   
  thick,
  main node/.style={},
]
  \newcommand*{\MainNum}{11}
  \newcommand*{\MainRadius}{3cm} 
  \newcommand*{\MainStartAngle}{90}
  % Print main nodes, node names: p1, p2, ...
  \path
    (0, 0) coordinate (M)
    \foreach \t [count=\i] in {1,...,11} {
      +({1-\i)*360/\MainNum + \MainStartAngle}:\MainRadius)
      node[main node, align=center] (species_\i) {\GetListMember{#1}{\i}}
    }
  ;  

  % Calculate the angle between the equal sides of the triangle
  % with side length \MainRadius, \MainRadius and radius of circle node
  % Result is stored in \p1-angle, \p2-angle, ...
  \foreach \i in {1, ..., \MainNum} {
    \pgfextracty{\dimen0 }{\pgfpointanchor{species_\i}{north}} 
    \pgfextracty{\dimen2 }{\pgfpointanchor{species_\i}{center}}
    \dimen0=\dimexpr\dimen2 - \dimen0\relax 
    \ifdim\dimen0<0pt \dimen0 = -\dimen0 \fi
    \pgfmathparse{2*asin(\the\dimen0/\MainRadius/2)}
    \global\expandafter\let\csname p\i-angle\endcsname\pgfmathresult
  }

  % Draw the arrow arcs
  \foreach \i [evaluate=\i as \nexti using {int(mod(\i, \MainNum)+1}]
  in {1, ..., \MainNum} {  
    \pgfmathsetmacro\StartAngle{   
      (\i-1)*360/\MainNum + \MainStartAngle
      + \csname p\i-angle\endcsname
    }
    \pgfmathsetmacro\EndAngle{
      (\nexti-1)*360/\MainNum + \MainStartAngle
      - \csname p\nexti-angle\endcsname
    }
    \ifdim\EndAngle pt < \StartAngle pt
      \pgfmathsetmacro\EndAngle{\EndAngle + 360}
    \fi
    \draw[<-]
      (M) ++(\StartAngle:\MainRadius)
      arc[start angle=\StartAngle, end angle=\EndAngle, radius=\MainRadius]
       node[midway,above,sloped] {\tiny{$#2$}};    
    ;
  }
\end{tikzpicture}}
%%END CIRCLE TOURS



\title{Projects}

\begin{document}
\begin{abstract}
    Some ideas for projects.
\end{abstract}
\maketitle

\newpage
\section{Musical $\pi$}


$\pi$ is an irrational number, meaning a decimal number that cannot be written as a fraction. It has an infinite amount of decimal digits and the patterns on the number never repeat.

There are other irrational numbers, like $\sqrt{2}$ or the golden ratio.

\begin{question}
     Choose your favorite irrational number and find online its first 50 digits. \end{question}
    \begin{question} Remember the 700 cents tour on the notes. The resulting notes were:
    \[
C \lto^{700} G \lto^{700} D \lto^{700} A \lto^{700} E \lto^{700} B \lto^{700} F\sharp \lto^{700} C\sharp\lto^{700} G\sharp \lto^{700} D\sharp \lto^{700} A\sharp \lto^{700} F \lto^{700} C
\]
    Assign a digit to each of those notes (there will be a couple unused ones). For example:
    \[
C \lto^{700} G \lto^{700} D \lto^{700} A \lto^{700} E \lto^{700} B \lto^{700} F\sharp \lto^{700} C\sharp\lto^{700} G\sharp \lto^{700} D\sharp \lto^{700} A\sharp \lto^{700} F \lto^{700} C
\] 
\hspace{15 pt}
    0 \hspace{18 pt} 1 \hspace{18 pt} 2 \hspace{18 pt} 3 \hspace{18 pt} 4 \hspace{18 pt} 5 \hspace{18 pt} 6 \hspace{22 pt} 7 \hspace{22 pt} 8 \hspace{22 pt} 9

This is just an example. You can choose your own key!

\textbf{Note:} Remember that playing the notes in piano order one after the other does not sound as good, which is why it is better to use the notes in the order of the 700 cents tour. Another example of a set of notes that sound good together is just the white keys: $C, D, E, F, G, A, B$.
\end{question}
   \begin{question} Write down the sequence of notes that would correspond to the digits of your irrational number according to your chosen key. 
\end{question}  
\begin{question}
You have pitch, now work with rhythm. You can start by making all the notes be quarter notes, but that wouldn't make an interesting piece. Decide the duration for each of the notes on your sequence. You can relate that to $\pi$ too. For example, since the first digit of $\pi$ is 3, the first note could have a duration of 3 eighth notes. Following this pattern, the second note, which corresponds to the digit 1, would be 1 eighth note. Remember this is just an example - choose your own key for the duration of the notes. \end{question}

\textbf{CHALLENGE (optional)} Decide on a time signature and write the music sheet for your composition.

\begin{question} Encode the result on the online sequencer and check how your irrational number sounds! 
    
\end{question}

\textbf{More variations} 

If you prefer, instead of an irrational number you can use your favorite sequence of numbers, like the Fibonacci sequence or the prime numbers. Then make each digit into a note. 

If you are familiar with modular arithmetic, you could also use mod 12 on each number on the sequence. Then your key would use all 12 notes.



\subsection{What to Submit}
Record a video explaining what you did: what irrational number and what key you used. What method you used for the rhythm. Play your encoded piece as well.


\newpage
\section{Tune and Play your One-Stringed Instrument}

On Thursday, we worked on building a one-string instrument out of a soda bottle. This project tries to make the most out of it and learn how to play actual songs on it.

In this camp, we learned that the frequency of a string is related to its length.  Since notes are different frequencies, we can get different notes with different string lengths.  But what if we have just one string?  We can change the "length" of our string by placing our finger on it so when it is plucked, it is not the whole string which vibrates.  This is the basic principle behind playing a guitar!

You may recall that a note one octave above another gets double frequency, which means the wavelength would be halved.  This means plucking a string with our finger placed halfway along the string should produce a note one octave above the original string length.

The ratios of the frequencies between the other (non-octave) notes are also known.

Other string instruments like the violin rely on the facts above to produce different sounds. We, too, will take advantage of these facts to tune our instrument.

\subsection{Materials}
\begin{itemize}
    \item DIY one-string instrument
    \item Yard long ruler
    \item Pencil
    \item Marker
    \item Tuning app or website
\end{itemize}

\begin{question} First, make sure the string on your instrument is tense. Measure the vibrating portion of the string in your instrument. This should mean measuring only one side right from where the bridge (bottle cap) ends to the groove on the end of the pipe on that same side. As you measure, be careful not to press on the string. You want to measure its actual length when not being played. Take note of this length.
\end{question}
\begin{question}
Get a tuning app in your phone or computer and pluck the string to identify the note that it plays. We suggest the app Tune!it Lite (android) or Pano Tuner (apple) as those were the apps we used when experimenting.  However, most chromatic tuner apps will work.
\end{question}

\begin{question}
Divide the length of the string by 2 and make a mark with a \textbf{pencil} directly on the \textbf{string}. Again, try not to press on the string as you measure.
\end{question}

\begin{question}
Pluck the string while holding it at this half way mark.  See if the tuner says the note is about an octave above the note of the full string (it might not be exact).
\end{question}

\begin{question}
Now we need to do some calculations. Compute the frequency ratio between your instrument's open string note and the six notes right above it. 
    
    For example, if the open string gives a C, I would compute the ratio between its frequency, 261.6, and the frequency of the next D, 293.7, to get:
\[
\frac{294}{262}\approx1.122
\]
And I would repeat this calculation with the frequency of the next $C\sharp, D\sharp, E, F, F\sharp, G, G\sharp, A, A\sharp$, and $B$.
    
    \noStaff
    
    

\textbf{Note:} You do not need to worry about getting the correct subindex. You should get the same ratio between, say $C_4$ and $D_4$ as with $C_5$ and $D_5$.

Record your calculations on the table below.
\end{question}

\begin{question}
Once you have all these values, you are ready to calculate the length of the string that has to vibrate to produce each note. Just like to mark an octave above we had to divide by 2, to calculate the rest of the lengths you will have to divide by the numbers you calculated.

For example, for an instrument that plays a C, to get the D we need to divide the length of the string by $1.122$.

Divide the length of your string by the ratios you calculated in the previous question and write down the corresponding length for each note.

Here is a table you can use to record your calculations and numbers for the previous two parts:\\
\\
\includegraphics[width=\textwidth]{ratio-notes-table.png}

Here is an example of how you might start filling it in based on the example calculations with C:\\
\\
\includegraphics[width=\textwidth]{example-table.png}
\end{question}
    
\begin{question}
Now measure and mark the lengths you calculated on your instrument's string, carefully aligning the 0 on the ruler with the upper end of the cap. Make a mark with \textbf{pencil} on the string for each length. Remember to try not to press on the string while measuring.
\end{question}

\begin{question}
Since there's some error in the calculations and the measuring, now we'll adjust the tuning with the tuner app. Press on the string putting your finger right on the first mark. It is better if you do not press all the way down, but instead slightly touch the string on the mark. Pluck the string and see what the tuner says. Move your finger slightly above and below the mark, plucking the string each time, until you find the right spot to get the note you were looking for. Once you are happy with the result, mark the final spot using a \textbf{marker}.

Repeat the process with all the notes.
\end{question}

\begin{question}
Your instrument is ready! The next goal is to practice and play \textit{Twinkle, Twinkle, Little Star} or another simple piece you like. 
\end{question}

\begin{question}
You might need to adjust the piece to your instrument range. Ideally, the note you start \textit{Twinkle, Twinkle, Little Star} on should be the note your instrument plays with the open string. If you need to adjust, remember you should be using the transposition method we learned to maintain the song's melody.
\end{question}

\subsection{What to Submit}

Record a video explaining what you did. Show your work as well as a performance of the piece.


\newpage

\section{Tune and Play a Glass Harp}

In this camp, we learned that the frequency of a string is related to its length.  Since notes are different frequencies, we can get different notes with different string lengths.  But the same principle is applied in other instruments, which is why smaller instruments make higher pitched sounds, and larger instruments make lower pitched sounds.  So we can use this same principle to make notes with other mediums as well, such as glasses!

The goal of this project is to investigate the relationship between the volume of liquid in the glasses and the frequency of the notes created with those various volumes.  Using this investigation, you can create a glass harp, also known as musical glasses. 

This project still requires some materials, but these are hopefully easier to get and assemble than the materials for the one-string instrument.

\subsection{Materials}
\begin{itemize}
    \item Wine glasses.* It would be great if you have two or more types, but it is not required.  It would also be great if you have several identical ones (like a set of 4, 6, or 8), but it is also not required.  Even if they are different types, having at least 6 glasses would be ideal as you need 6 notes in order to play \textit{Twinkle, Twinkle, Little Star}.  But it is possible to complete this project with just a single glass (it will just be harder).
    \item Water
    \item Measuring cups
    \item A straw or dropper
    \item A chromatic tuner app in your phone or computer.  We suggest the app Tune!it Lite (android) or Pano Tuner (apple) as those were the apps we used when experimenting.  However, most chromatic tuner apps will work.
\end{itemize}

\textbf{Note:} You want wine glasses as it is harder to play ``normal'' glasses by rubbing your finger along the rim.  However, if you can get this to work with normal glasses, then feel free to use those! Pro-tip: glasses with thinner walls will work better.

\subsection{Learn to Play the Glasses}

Here's how to ``play'' the wine glasses: Moisten a finger and run it lightly around the rim of a glass. Practice with a lighter or firmer touch until you find the correct level of pressure that produces a nice sound. The type of glass and the quality of water also affect the sound.

This video might help you figure out how to get your wine glasses to play a note:\\
\url{https://youtu.be/AxWzVPdubjs}
\\
\\
Keep in mind that you need to keep your finger moist, so as you play you might have to dip it in water several times. You might also want to tape the glasses to the table, to make sure you don't accidentally tip one over while playing.\\
\\
Try practicing until you feel pretty comfortable making your glasses ``sing'' before you continue!

\subsection{Gather Data}
\begin{enumerate}

\item Pick one glass.  Determine its volume by filling it up all the way to the top, trying to stop right before the water spills. Carefully empty the water into the measuring cup and take note of the volume.

\item Now calculate and record the various volumes to fill the glass with and play and record the note for each volume:\\
\includegraphics[width=\textwidth]{Volume.png}\\
\\
If you have several identical glasses, you can fill each glass with a different amount and then play the notes.  If you have just one version of a glass, then you can fill it to a certain volume, play the note, record it and then empty it to fill it with a different amount of water.

\item \textbf{Optional:} Repeat! If you have several different types of glasses (like two different types of wine glasses, or even more!) then repeat this step for each glass type!\\
Here are two extra tables for recording:\\
\includegraphics[width=\textwidth]{Volume.png}\\
\\
\includegraphics[width=\textwidth]{Volume.png}\\
\\
Feel free to do more types of glasses as well (by just writing out your own table on another sheet of paper).

\end{enumerate}


\subsection{Questions to Answer}
\begin{enumerate}

    \item If you play a note in a glass, and then add some more water to it and play another note, what is happening to the pitch of the notes?
    
    \item Suppose there is not a lot of water in your glass.  You add a \textit{little} bit of water to it.  Will the frequency of the new note be very different or very similar to the note from before you added water?  Explain how you know this from the data you collected.
    
    \item Suppose there is a lot of water in your glass.  You add a \textit{little} bit of water to it.  Will the frequency of the new note be very different or very similar to the note from before you added water?  Explain how you know this from the data you collected.
    
\end{enumerate}


\subsection{Find \textit{Twinkle, Twinkle, Little Star} Notes}
\begin{enumerate}
    \item Now we're going to prepare our glasses for playing \textit{Twinkle, Twinkle, Little Star}.  You will either need 6 glasses (identical ones will be easier to work with, but they do not need to be), or, if you do not have 6 glasses to play, you can use just one glass.  However, you will want to record yourself playing \textbf{each} note after you find it so that you can use that recording to splice together a piece of you playing the song.
    
    \item Identify the notes you will use.  This will depend on your glasses and their note range.  Normally, \textit{Twinkle, Twinkle, Little Star} starts on a "C."  However, recall that we can start on any note we would like and it will still sound like the same song as long as we keep the cents difference between notes the same.  
    
    Choose the lowest of your glass notes and use the transposition method we learned to write the notes of \textit{Twinkle, Twinkle, Little Star} so that it starts on that lowest note. (You can choose the second or third lowest if your glasses cover a big range.)
    
    Record the notes on your transposed \textit{Twinkle, Twinkle, Little Star} and identify their frequencies.
    
    %So determine from your glasses and from how you'd like to play \textit{Twinkle, Twinkle, Little Star} what 6 notes you will need to play and identify their frequencies which fall within the range of your glasses:\\
    \includegraphics[width=\textwidth]{Projects/Twinkle-Freq.png}\\
    
    \item Compare the frequencies on your piece with the frequencies you got from your glasses. Choose the 6 glasses/filling levels that come closest to your song's notes.
    
    \item Take your 6 glasses and fill them to the amount that gets you closest to your desired note.  Then start to add or remove water to get to the desired note.  Depending on the water amount and the needed frequency, you may want to start by added/removing a little, or a lot of water.  If add/remove just a small amount, using a straw or small spoon to add/remove small amount may make the process easier.
    
    You might want to mark your glasses with the name of the note, so you remember which one is which.
    
    \item Remember, if you do not have 6 glasses and are using only 1 glass, make sure you record yourself playing the desired note before you change the volume to start finding a different note!
    
\end{enumerate}


\subsection{Play \textit{Twinkle, Twinkle, Little Star}}

\textit{If you have 6 glasses:}  Now that you have your glasses filled to give you the correct notes for \textit{Twinkle, Twinkle, Little Star}, record yourself playing it and include it with your other project work!

\textit{If you have fewer than 6 glasses:} Hopefully you recorded yourself playing each note as you found them with your glasses in the previous section.  Now use a computer or other editing device to splice together your recording of you playing each note in the correct order of \textit{Twinkle, Twinkle, Little Star} and include this spliced together recording with your other project work!\\


\subsection{Useful Links}
Check this video for a demonstration playing the glass harp:

\link[Singing wine glasses - At home science - ExpeRimental #4: https://youtu.be/FNTmzX3GpOU]{https://youtu.be/FNTmzX3GpOU}

\subsection{What to Submit}

Record a video explaining what you did. Show your work as well as a performance of the piece.




\newpage
\section{The Pythagorean Scale and Equal Temperament}

For this project, make sure you have completed the first two tasks for the "Strings and Overtones" topic.

This project takes you through some calculations to explain why there are 12 tones and learn about Pythagorean tuning vs Equal Tempered tuning.

%Imagine you are a philosopher from Ancient Greece and you are trying to put some structure into music. You have decided that you want to design a system of notes, much like the alphabet, but for music. You know that music is made of sound waves, and that each frequency corresponds to a different sound. Rather than dealing with an infinite number of sounds coming from different frequencies, you think it would be a good idea to choose a few selected frequencies, give them a name, and combine these few in different ways to make music.

There once was a philosopher from Ancient Greece who was experimenting with music. He took a string that is taut and fixed in both ends and play it. He loved fractions so naturally he wonder what would happen if he used a string that is half the length of the first one. He played it and...

\begin{question}
 What did he get?
 \end{question}

He thought that was pretty neat so he wonder what would happen if he continued with his fractions, meaning, what if he cut the string to $\frac{1}{3}$ of the original length. And then $\frac{1}{4}$, and then $\frac{1}{5}$ and so on.


\begin{question}  Check back on your work from Strings and Overtones (Thursday). Let  $L$ be the wavelength of a string. 

\begin{enumerate}
    \item Write a formula to calculate the wavelength of a string of length $\frac{1}{n}$ the length of the original string.
\begin{hint}
Start with some easier examples like if you cut the string in half or to a third.
\end{hint}

\item Now let $f$ be the frequency of that same string. How is this frequency changing as we cut the string to a fraction of it? Write a formula that relates the frequency of the original string $f$ with the frequency of a string $\frac{1}{n}$ the original length.
  \end{enumerate} 
  \end{question}
\begin{question} Suppose that the note the philosopher started at is $C_4$, which has a frequency of about 261.6 Hz. Use this value to figure out what frequencies the philosopher got from cutting the length as described before. Record them in a table like this one.

\begin{center}
\renewcommand{\arraystretch}{3}
\begin{tabular}{|c|c|c|c|c|c|c|c|c|c|c|c|c|}\hline
Fraction of length & Original (1) & $\frac{1}{2}$ & $\frac{1}{3}$ & $\frac{1}{4}$ & $\frac{1}{5}$  & $\frac{1}{6}$ & $\frac{1}{7}$ & $\frac{1}{8}$ & $\frac{1}{9}$ & $\frac{1}{10}$ & $\frac{1}{11}$ & $\frac{1}{12}$    \\\hline
Frequency & &&&&&&&&&&& \\\hline
\end{tabular}
\end{center}
\end{question}
\begin{question} You will notice that he got pretty high frequencies. To be able to work with this more easily, let's bring all the notes down within a single octave.

Take each frequency (except the first two) and bring it one or more octave downs until you find a value that is between $C_4$ and $C_5$ (261.6 Hz to 523.2 Hz). Remember from the Hertz task how to calculate the frequency of a note that is one octave down.

Add a row to you table and record these values too.

\begin{center}
\renewcommand{\arraystretch}{3}
\begin{tabular}{|c|c|c|c|c|c|c|c|c|c|c|c|c|}\hline
Fraction of length & Original (1) & $\frac{1}{2}$ & $\frac{1}{3}$ & $\frac{1}{4}$ & $\frac{1}{5}$  & $\frac{1}{6}$ & $\frac{1}{7}$ & $\frac{1}{8}$ & $\frac{1}{9}$ & $\frac{1}{10}$ & $\frac{1}{11}$ & $\frac{1}{12}$    \\\hline
Frequency & &&&&&&&&&&& \\\hline
Octave(s) Down & &&&&&&&&&&&\\
Frequency & &&&&&&&&&&& \\\hline
\end{tabular}
\end{center}
\end{question}
\begin{question} Look back to the piano image we used on the Hertz task. Find the philosophers frequencies. Which notes did the philosopher get with his method? (Note that you won't find the exact values, but good approximations.)
\end{question}

\begin{question} 
Let's now think a little more abstract, considering that this process would have worked similarly, regardless of the note that the philosopher's original string played. Since we are imagining that we don't know what this frequency is, let's call it $f$. To get the frequency of the note produced by a third of the length you should have multiplied by 3, but the result would have been beyond the range we wanted, so you then divided by 2. All in all, the frequency of this note is $\frac{3}{2}f$. Following this example, write down the fractions by which you ended up multiplying the original frequency to get the other notes.

Note that these fractions should be:
\[
\dfrac{\text{the number you multiplied by to obtain the frequency of the fraction string}}{\text{the number you divided by to bring down as many octaves as needed}}
\]

Write these fractions in ascending order.

\end{question}


What you worked on above is one way of deriving the 12 note scale used in Western music, commonly attributed to Pythagoras, although others had worked on the idea before. Let's now look at another way of deriving it.

%The question is how to go about selecting those notes. You are a wise one and know better than choosing at random, so you decide to look for a more thorough method. You think: "Let's say I already chose one note and I want to find other notes that make a good system with that first one. I'm still not sure how I will choose the first one, so for the moment, I'm just going to say that its frequency is $f$."

Again, we start with a note and we will call its frequency by $f$. As you know, the same note an octave up would be $2f$. The next note that Pythagoras obtained was $\frac{3}{2}f$:
    
%You then play a note with twice the frequency, meaning $2f$, the second harmonic, and realize that it sounds pretty similar to the first one.

%\textbf{Remember your work from this week: What is the relation between a note with a  given frequency and the note with twice that frequency?}
    
%But this note sounds too similar to the first.  You want to find something that combines well but does not sound so similar. So, you do the obvious thing, you use the next harmonic, the third, multiplying the frequency by 3. A note with a frequency of $3f$ is too far from the original $f$, though. It would have a pitch higher than $2f$. Since $f$ and $2f$ sound so similar, you think it would make more sense to choose all your notes between these two frequencies. So instead of using $3f$, you select the note one octave below: $\frac{3f}{2}$. So, now you have 3 notes with the following frequencies:

\[ f \qquad \frac{3f}{2} \qquad 2f
\]
    
%So far so good. You play the notes and think this $\sfrac{3}{2}$ factor seems to work well, so you state the formal idea of your music scale:

The idea is to create a scale by successively applying this factor of $\frac{3}{2}$ and bringing down an octave when necessary to stay within a single octave as before, between $f$ and $2f$.

For example, to get the next note, we would do:


%Following your shiny new rule, you now use this factor of $\frac{3}{2}$ on the last note you added to the scale, the one that has a frequency of $\frac{3}{2}f$:

\[
\frac{3}{2}f\times\frac{3}{2}=\frac{9}{4}f
\]

Since $\frac{9}{4}f$ is not between $f$ and $2f$, as $\frac{9}{4}>2$, we need to bring the note down an octave by dividing it by $2$.  At this point, the scale is:

\[ f \qquad \frac{9}{8}f \qquad \frac{3}{2}f \qquad 2f
\]




\begin{question} Now you try! 
\begin{enumerate}
\item To continue with this process, take the last note we created and multiply it by the chosen factor: $\frac{3}{2}$.
    \item Check if the resulting number is between $f$ and $2f$. If it isn't, then bring it down an octave. This is the new note.
    \item Repeat the process a couple more times, until you have seven notes.
    \end{enumerate}
\end{question}    
    
 \begin{question}   All of the notes above were added by multiplying a note on the list by $\sfrac{3}{2}$, except $2f$. That one was just added by raising an octave. Would it be possible to add a note to the scale so that its third harmonic is the note with frequency $2f$?
    
 Find the frequency of a note such that, when multiplied by $\frac{3}{2}$ gives $2f$. That is the last note on the scale. 
\end{question}     
\begin{question}Organize the scale by writing the fractions in order, as we were doing above. You should have eight notes total, including $f$ and $2f$.
\end{question} 
    
\begin{question} For each note on the scale, calculate its ratio with the previous note. 
    \begin{hint} Most of the ratios should be equal. In music, this is called a \textit{major second} or a \textit{whole tone}. The other ratio you will find, which only repeats twice, is a semi-tone.   \end{hint}
\end{question}     
    
\begin{question}Remember that we created this scale in the abstract, without choosing a specific note to start with. Let's now finalize our scale by choosing $C_4$ again as our base note and calculating the frequencies of all the notes on the scale. (Basically, substitute $f$ in your scale with 261.6 Hz.)\end{question} 
    
 The problem with this scale is that if you choose another note on the scale as your new starting note, most of the notes will be the same, but not all of them. Hence, you end up with a bunch of different scales. 
 %The original idea of using a simple ratio like $\sfrac{3}{2}$ was to get simple, easy to use ratios and produce a very regular scale.
 
\begin{question}  Use the frequency of $G_4$ as your new $f$ and compute the frequencies of the scale with this new $f$. Compare with the values you obtained in the previous question. Do you get the same notes?\end{question} 
 
\textbf{Bonus:} Can you pin point what in the previous questions makes it obvious that we won't get the same notes even if we start on one of the notes on the $C$ scale?
 
By repeating this process, you would end up with 13 different scales and many different notes, although for some, their frequencies are very close, so they are heard as the same note. By reducing notes that are very similar, music theory ended up with 12 notes.


\begin{question} Compare the fractions on this scale (on part 24) with the fractions on the first scale. Which fractions on one scale are close to a fraction on the other?
\begin{hint}
Convert the fractions to decimal values so you can compare more easily.
\end{hint}
\end{question} 

Fast forward to the sixteenth century and someone thought that, to avoid the issue of multiple scales, it might be a better idea to split the octave interval in 12 equal intervals. You might be thinking this would be easy to do by simply dividing by 12, but remember frequency does not grow linearly, but exponentially. We need multiplicative, not additive, intervals. This is just as easy as dividing by 12 though, at least with a calculator. The interval needed is:

\[\sqrt[12]{2}\approx1.059
\]
This 12th root can also be written as $2^{\frac{1}{12}}$.

What this means is that the frequency of the notes in this scale, called \textit{equal temperament}, are related as follows:

\[
A \lto^{\times\sqrt[12]{2}} A\sharp \lto^{\times\sqrt[12]{2}} B \lto^{\times\sqrt[12]{2}} C \lto^{\times\sqrt[12]{2}} C\sharp \lto^{\times\sqrt[12]{2}} D \lto^{\times\sqrt[12]{2}} D\sharp 
\lto^{\times\sqrt[12]{2}} E
\lto^{\times\sqrt[12]{2}} F
\lto^{\times\sqrt[12]{2}} F\sharp
\lto^{\times\sqrt[12]{2}} G
\lto^{\times\sqrt[12]{2}} G\sharp
\]

What this means is, if we again denote the frequency of $C$ as $f$:
\[
\text{Frequency of } G =\left(2^{\frac{1}{12}}\right)^7f=2^\frac{7}{12}f\approx 1.498f
\]

where the 7th power comes from the fact that there are 7 jumps between C and G (700 cents!), so you have to multiply $2^{\frac{1}{12}}$ by itself 7 times to go from the frequency of one note to the other.

\begin{question} Repeat the process we just did with G but for all other non-sharp notes, always comparing against C. Contrast the resulting values with the ratios in the previous scale.
\end{question} 

Notice that because all the jumps or ratios between notes on the equal temperament scale are equal, it doesn't matter what we use as our starting frequency, we will always get the same set of notes (and their octaves, of course). This is the great advantage of this system, but it it less intuitive than the Pythagorean, since $\sqrt[12]{2}$ is an irrational number and it does not relate with harmonics as did the $\frac{3}{2}$ ratio.



\subsection{Useful Links}

\link[Why It's Impossible to Tune a Piano by Minute Physics - https://youtu.be/1Hqm0dYKUx4]{https://youtu.be/1Hqm0dYKUx4}
\link[Why are There Twelve Notes in Music by Steven Jacks - https://youtu.be/IT9CPoe5LnM]{https://youtu.be/IT9CPoe5LnM}


\subsection{What to Submit}
Record a video explaining what you did and what you learned on this project. Be sure to show your work!


\newpage



\begin{question}
\textbf{Musical Magic Square}
\begin{instructorNotes}
\url{http://www.frommusictomath.com/uploads/4/3/3/2/43320843/daviesmagicsquares.pdf }
\end{instructorNotes}

\begin{enumerate}
    \item Complete the following magic square so that it has all the numbers 1 to 64.
    
    \begin{center}
    \renewcommand{\arraystretch}{2}
\begin{tabular}{|c|c|c|c|c|c|c|c|}\hline
8 &58 &59 & 5 & 4 &62 & 63 & \hspace{20 pt} \\\hline
49 & \hspace{20 pt} & \hspace{20 pt} & 52 & 53  & 11 & 10 & 56 \\\hline
41 & 23 & \hspace{20 pt} &44 & 45 & 19 & 18 & 48 \\\hline
\hspace{20 pt} & 34 & 35 & 29 & \hspace{20 pt} & 38 & 39 & 25 \\\hline
40 & 26 & 27 & 37 & 36 & 30 & 31 & 33 \\\hline
17 & 47 & 46 & 20 & 21 & \hspace{20 pt} & 42 & 24 \\\hline
9 & 55 & 54 & 12 & 13 & 51 & \hspace{20 pt} & 16 \\\hline
64 & 2 & 3 & \hspace{20 pt} & 60 & 6 & 7 & 57 \\\hline
\end{tabular}
\end{center}

\item Now consider the following sequence of notes and transpose it so that it starts on the notes indicated on the first column. Fill in the table with your transposed sequences.

\[G \qquad E \qquad F \qquad D \qquad F\sharp \qquad A \qquad G\sharp \qquad C
\]

Remember that transposing means "moving" the song so that it starts on a new note, but so that the cent difference between adjacent notes is preserved.
\begin{hint} Transposing is easier to do on the online sequencer!!!
\end{hint}

\begin{center}
    \renewcommand{\arraystretch}{2}
\begin{tabular}{|c|c|c|c|c|c|c|c|}\hline
$G$ & $E$ & $F$ & $D$ & $F\sharp$ & $A$ & $G\sharp$ & $C$\\\hline
$E$ & \hspace{20 pt} & \hspace{20 pt} & \hspace{20 pt} & \hspace{20 pt} & \hspace{20 pt} & \hspace{20 pt} & \hspace{20 pt} \\\hline
$F$ & \hspace{20 pt} & \hspace{20 pt} & \hspace{20 pt} & \hspace{20 pt} & \hspace{20 pt} & \hspace{20 pt} & \hspace{20 pt} \\\hline
$D$ & \hspace{20 pt} & \hspace{20 pt} & \hspace{20 pt} & \hspace{20 pt} & \hspace{20 pt} & \hspace{20 pt} & \hspace{20 pt} \\\hline
$F\sharp$ & \hspace{20 pt} & \hspace{20 pt} & \hspace{20 pt} & \hspace{20 pt} & \hspace{20 pt} & \hspace{20 pt} & \hspace{20 pt} \\\hline
$A$ & \hspace{20 pt} & \hspace{20 pt} & \hspace{20 pt} & \hspace{20 pt} & \hspace{20 pt} & \hspace{20 pt} & \hspace{20 pt} \\\hline
$G\sharp$ & \hspace{20 pt} & \hspace{20 pt} & \hspace{20 pt} & \hspace{20 pt} & \hspace{20 pt} & \hspace{20 pt} & \hspace{20 pt} \\\hline
$C$ & \hspace{20 pt} & \hspace{20 pt} & \hspace{20 pt} & \hspace{20 pt} & \hspace{20 pt} & \hspace{20 pt} & \hspace{20 pt} \\\hline
\end{tabular}
\end{center}

\item Number the boxes on the note chart, left to right and top to bottom.

\item Map the notes into the magic square. In other words, substitute each number on the magic square by its corresponding note according to the note chart.

\begin{center}
    \renewcommand{\arraystretch}{3}
\begin{tabular}{|c|c|c|c|c|c|c|c|}\hline
\hspace{30 pt} & \hspace{30 pt} & \hspace{30 pt} & \hspace{30 pt} & \hspace{30 pt} & \hspace{30 pt} & \hspace{30 pt} & \hspace{30 pt} \\\hline
&&&&&&&\\\hline
&&&&&&&\\\hline
&&&&&&&\\\hline
&&&&&&&\\\hline
&&&&&&&\\\hline
&&&&&&&\\\hline
&&&&&&&\\\hline
\end{tabular}
\end{center}

\item Choose 8 different instruments on the online sequencer. Your first instrument will play the first row on the chart above. Your second instrument will play the second row on the chart, and so on. For each instrument, encode the sequence of notes on the corresponding row.
\begin{hint}
You can play several windows of the online sequencer at the same time.
\end{hint}

\item Make another version of the magic square. This time, replace each number by is remainder when dividing by 8. For example, 15 would change to 7. For the numbers with a 0 remainder (multiples of 8), use 8 instead of 0.

\begin{center}
    \renewcommand{\arraystretch}{3}
\begin{tabular}{|c|c|c|c|c|c|c|c|}\hline
\hspace{30 pt} & \hspace{30 pt} & \hspace{30 pt} & \hspace{30 pt} & \hspace{30 pt} & \hspace{30 pt} & \hspace{30 pt} & \hspace{30 pt} \\\hline
&&&&&&&\\\hline
&&&&&&&\\\hline
&&&&&&&\\\hline
&&&&&&&\\\hline
&&&&&&&\\\hline
&&&&&&&\\\hline
&&&&&&&\\\hline
\end{tabular}
\end{center}

\item Consider that the numbers above are the duration of each note in eighths. For example, the first cell above shows an 8, meaning the note on that same position, a $C$ would last 8 eighth notes, or one whole note.

Modify the note duration on your encoded passages to match the chart above.

\end{enumerate}
\end{question}




\end{document}