\documentclass[12pt]{ximera}

\title{Introduction}

\begin{document}
\begin{abstract}
    We describe our goals.
\end{abstract}
\maketitle



Mathematics and music are two areas that initially seem completely unconnected. It takes work to do math, and most don't find it enjoyable. On the other hand, music is easily enjoyed passively.


However, music is an arrangement of rhythm and pitch over time into patterns. Once we start to study the production of music, we find many mathematical ideas. It soon is apparent that mathematics is everywhere in the production of music, leading some to sing the anthem
\begin{center}
    ``music is how math sounds.''
\end{center}
But the actualization of this is too often left as an exercise for the reader. While mathematics can be used to describe music \ref{math-music-books}, and mathematical thinking can be inspire music \ref{giantsteps}, how does music give back to mathematics?

I'll tell you a secret. Many questions in mathematics come in two versions, a discrete version, and a continuous version. 

Give some eg.
Discrete: data points, whole numbers
Continuous: linear models, numbers between 0 and 1

Music is interesting in that the concepts of music are simultaneously continuous and discrete. The classical theory of western musical harmony (there are others!), is an essentially abstract theory of how pitches sound together. It attempts to answer when pitches will be harmonious (sound pleasant together) or 
dissonance (the opposite). In some sense it is a ``chemistry'' of music. 


The structure of this theory of harmony is based on fractional relations between wavelengths. However, discrete, essentially number-theoretic basis for harmony, does not yield the abstract theory it inspired!


Instead we reanalyze, and turn to the continuous mathematics of analysis of exponents to arrive at a new theory, however imperfect, that bridges the gap between our abstract theory and our discrete fractional understanding of harmony. 


After this we will dig even deeper, to understand the basic building blocks of rhythm and pitch. 
In particular, as we will see, we have dual notions:
\begin{center}
\begin{tabular}{c|c}
 Continuous    & Discrete \\ \hline\hline
   Frequency  & Notes \\
   Key & Beats\\
   String & Finger \\
   Waves &  Beeps \\
   Chord & Polyrhythm
\end{tabular}
\end{center}


As we complete the activities that follow, we will see exactly how the concepts of music are simultaneously ``continuous'' and ``discrete.'' Moreover, we will use physical properties of music to morph one viewpoint into the other. Let's be explicit, we're going on a musical, and mathematical journey. We will describe music at a basic level using mathematics, and by the end we will see:
\begin{itemize}
    \item Music can be described in terms of discrete mathematics.
    \item Music can be described in terms of continuous mathematics.
    \item We will give ways to go from one description to another.
\end{itemize}




Oh, and we'll learn a lot of math along the way!




\end{document}